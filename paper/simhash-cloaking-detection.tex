% TEMPLATE for Usenix papers, specifically to meet requirements of
%  USENIX '05
% originally a template for producing IEEE-format articles using LaTeX.
%   written by Matthew Ward, CS Department, Worcester Polytechnic Institute.
% adapted by David Beazley for his excellent SWIG paper in Proceedings,
%   Tcl 96
% turned into a smartass generic template by De Clarke, with thanks to
%   both the above pioneers
% use at your own risk.  Complaints to /dev/null.
% make it two column with no page numbering, default is 10 point

% Munged by Fred Douglis <douglis@research.att.com> 10/97 to separate
% the .sty file from the LaTeX source template, so that people can
% more easily include the .sty file into an existing document.  Also
% changed to more closely follow the style guidelines as represented
% by the Word sample file. 

% Note that since 2010, USENIX does not require endnotes. If you want
% foot of page notes, don't include the endnotes package in the 
% usepackage command, below.

% This version uses the latex2e styles, not the very ancient 2.09 stuff.
\documentclass[letterpaper,twocolumn,10pt]{article}
\usepackage{usenix,epsfig,endnotes}
\usepackage{graphicx,calc,subfigure,caption,float}
\usepackage[breaklinks,colorlinks]{hyperref}
\usepackage{endnotes,microtype,xspace,fancyvrb,multirow}
\usepackage{array,underscore,relsize}
\usepackage{booktabs,amsmath}
\usepackage{authblk,tabularx}

\input{cmds}

\begin{document}

%don't want date printed
\date{}

%make title bold and 14 pt font (Latex default is non-bold, 16 pt)
\title{\Large \bf Cloaking Detection through Simhash-based Website Model and
Crowdsourcing}
% Simhash-based Website Model: Towards Real-Time Cloaking Detection}
% Real-Time Cloaking Detection based on Simhash and Crowdsourcing

%for single author (just remove % characters)
\author{
}

%\author{
%{\rm Ruian Duan}\\
%\thanks{Georgia Institute of Technology}
%\and
%{\rm Weiren Wang}\\
%\thanks{Georgia Institute of Technology}
%\and
%{\rm Ji Zhang}\\
%\thanks{Google Inc.}
%\and
%{\rm Wenke Lee}\\
%\thanks{Georgia Institute of Technology}
%
%%wenke@cc.gatech.edu
%% copy the following lines to add more authors
%% \and
%% {\rm Name}\\
%%Name Institution
%} % end author
%



%\author[*]{Ruian Duan}%\thanks{ruian@gatech.edu}}
%\author[*]{Weiren Wang}%\thanks{weirenwang@gatech.edu}}
%\author[**]{Ji Zhang}%\thanks{jiz@google.com}}
%\author[*]{Wenke Lee}%\thanks{wenke@cc.gatech.edu}}
%\affil[*]{Georgia Institute of Technology}
%\affil[**]{Google Inc.}

\maketitle

% Use the following at camera-ready time to suppress page numbers.
% Comment it out when you first submit the paper for review.
\thispagestyle{empty}


\subsection*{Abstract}
Cloaking, used by spammers for the purpose of increasing the visiting rates of
their website, as well as to circumvent censorship, has been a challenging
spamming technique to search engines. Recently, there is a rising trend of
employing
cloaking in advertisement spam ~\cite{li2012knowing}.  The motivation of
cloaking 
is to hide the true nature of a website by delivering different semantic
content users versus censors.
Cloaking is popular due to low cost to setup and lack of efficient technique
to detect.

There are two major challenges in cloaking detection (A) revealing the
uncloaked content 
(B) differentiating cloaking from naturally dynamic changes of the same page.
This work employs
simhash-based website model to detect cloaking. The key idea to address
challenge 
(A) is to crowd-source data collection to user. However, collect such
information 
introduces large volume of traffic and violates user privacy. 
This work proposes simhash-based website model to address challenge (B)
with minimum amount of data and guarantee of user privacy.

In order to show the effectiveness of simhash-based website model, we build
a prototype and results 
show that we can achieve more than 95\%? true positive rate with 5\%? false
positive rate.





\section{Introduction}
\label{s:intro}

Cloaking, used by spammers for the purpose of increasing the visiting rates of
their website, as well as to circumvent censorship, has been a challenging
spamming technique to search engines. Recently, there is a rising trend of
employing cloaking in advertisement spam~\cite{li2012knowing}.  The motivation
of cloaking is to hide the true nature of a website by delivering different
semantic content users versus censors.

Spammers use user agent, referer, IP etc based cloaking techniques to
differentiate censors and normal users.

Cloaking is popular due to the low cost to setup and lack of efficient technique
to detect. Spammers use user agent, referer, IP etc based cloaking techniques to
differentiate censors and normal users.

There are two major challenges in cloaking detection (A) revealing the uncloaked
content (B) differentiating cloaking from naturally dynamic changes of the same
page. Existing approaches pretend to be the user, e.g. use proxies, set referer,
to reveal the uncloaked content, but spammers could detect and block them. An
intuitive way is to collect data from user directly. However, it is hard to
collect sufficient data for comparison while protecting user privacy.


The workflow is to collect page contents simhash on the user side, and compare
them to simhash of the same link from ad serving company to find cloaking. When
the differences of the simhashes are significantly large, the page is marked
cloaking. We generate two simhash for page content and structure respectively.
Intuition behind this is, simhash difference between different sites are larger
than different visits of the same site. We build a two-phase system to detect
cloaking: cluster learning phase, and cloaking detection phase. In the cluster
learning phase, an ad company visit urls and generate simhash from its content
with its owned IP, and learn pattern and distribution of the simhashes, i.e.
simhash-based website model. In the cloaking detection phase, the ad company
collects simhash from its users. Compare them with learned patterns, return
cloaking score or mismatch. Results on have shown data set show that we can
achieve more than 95\% True positive rate with 5\% false positive rate for page
content simhash and 100\% true positive rate with 2\% false positive rate for
structure simhash. If we define cloaking as mismatch in both page content
simhash and structure simhash (intersection of mismatches in both case), we can
achieve more than 99\% true positive rate with 1\% false positive rate.


This paper makes the following contributions:
(1) Introduce simhash-based website model to represent the content and layout
dynamics of a page.
(2) Propose the idea and framework to detect cloaking and protect user through crowdsourcing,
with low traffic overhead and privacy guarantee for user.
(3) Use of simhash-based website model, in cloaking detection, with better FPR
and TPR.
(4) Detect cloaking in both SEO and SEM , showing the seriousness of cloaking
problem in SEO and SEM.
(5) Examine the cloaking strategies employed by attackers.


The remainder of this paper is structured as follows. Section
~\autoref{s:related-work} provides a
technical background on cloaking and simhash, and related work in cloaking
detection. Section ~\autoref{s:swm} introduces the simhash-based website model.
Followed by a description of cloaking detection framework design in 
Section ~\autoref{s:framework}. Section ~\autoref{s:implementation} describes the 
prototype that we implemented and Section ~\autoref{s:experiment} shows our results,
followed in  Section ~\autoref{s:discussion} by a discussion of attack
model and defenses.



More fascinating text. Features\endnote{Remember to use endnotes, not
footnotes!} galore, plethora of promises.\\





\section{Background}
\label{s:related-work}
Search engine, as a most innovative technology introduced in late 20th century, has widely influenced our relationships. Search engine used their own page ranking algorthim to rank and indexed websites. In this way, users could find information by entering the search terms into the search enginee. To get better rankings and accurate indexed on search engines such as Google, websites are encouraged by Google to optimize their content. This is a technology called search engine optimization (SEO).
SEO policies were introduced by search engine later to maintain a fairness of search enginee ranking. Recently, some websites break the SEO policies to get better rankings on websites because of the large profits. These technology are called Black Hat SEO, which is used to get better rankings on search engine but break the SEO policy.
From the SEO policy, one of the most efficient methods to get better page rankings is to update the content of website frequently~\cite{wang2011cloak}.
Following this rule, cloaking, a Black Hat SEO method, is invented. Cloaking serves a blatently different information to users and Google to maintain their
profits and get better page rank. To be clear, we give a general example. Websites provide a frequently updated information to the Google so that they could get
better rank. On the other side, they send a different information to users to get profits.

\subsection{Example}
%To concrete the idea of cloaking, a specific cloaking example, malicious downlowd cloaking, is introduced. We entered the term "loose slot machines las vegas"
%in Google. Google returned a set of search results as showned in figure 1.
%From the figure 1, we see that the top 6 search results are all under domain "dje,.com.ol". If we disguised ourselves as normal users without setting HTTP proxy 
%agent and clicked the links, the broswers sent an HTTP request with Google reference. The website received our HTTP request and found Google reference. The website
%redirected several times, eventually landed to a page and automatically download malicious software. On the other hand, we reconfiguerd our browsers and set
%Google-Bot as our HTTP agent. We revisited the website. This time, the website provide us totally different information without redirecting and malicious downloading.
%From two scenarios, the cloaking idea is clear. Websites provide legal information to Google to get good page rank and avoid censorships. On the other side,
%websites provide different information with redericting and malicious downloads to get profits from users. 


To concrete the idea of cloaking, a specific cloaking example, traffic sale 
cloaking, is introduced. We entered the term "instant loan quote"
in Google. Google returned a set of search results as showned in
~\autoref{seo:search}, and the third result is doing cloaking. 
When clicking this link directly, ~\autoref{seo:user} is presented. However, if
we visit the landing page as spider (set user agent to Google bot),
~\autoref{seo:google} is shown.
%However, the content is actually from ~\autoref{seo:victim}. 
The fact is that, this website is doing blackhat SEO,
raising its rank with junk content, but when user clicks, they simply iframe
another page, monentizing from its ranking and the traffic.

%\subsection{Cloaking and Redirection}
%~\cite{wu2005cloaking} classifies 
%The landing page is different. The redirection.
%
%We use a different approach. We first click on the landing page, record the URL,
%and then disguise ourselves to visit that landing page. We leverage the fact
%that, if the cloakers employ some complex infrastructure to redirect users to
%different landing pages, and they are doing bad things. They will do user agent
%cloaking on the landing page as well. Otherwise, if search engine visits.
%
%This example: user goes to 'sh.st/pZ78x'. Google visit the same clickstring is
%redirected to '
%http://apktribe.com/games-andriod/slot-machines-space-of-a-laugh-las-vegas-on-line-casino-video-games-loose-spin-win-slots-roulette/'.
%However, if Google visit 'sh.st/pZ78x', Google is redirected back to
%'http://apktribe.com/games-andriod/slot-machines-space-of-a-laugh-las-vegas-on-line-casino-video-games-loose-spin-win-slots-roulette/'
%


\subsection{Search Engine Marketing}
Priviously, we briefly talked about search engine optimization. In this section, we introduce another area called search engine marking(SEM)
where enourmous cloaking websites exist. According to wikipedia~\cite{sem-wiki}, the term SEM is used to mean pay per click advertising,
particularly in the commercial advertising and marketing communities which have a vested interest in this narrow definition. To be clear,
search egnine marketing defines the advertisements shown on search engines after searching terms. Accoding to wikipedia~\cite{sem-wiki},
boundary beween SEO and SEM are sometimes not clear. However, when studying the cloaking, the SEO and SEM are clearly different. The difference
betwen them is because that the policies and rules on SEO and SEM are different. On SEM, the policy are usually strict and commertial-related. 
Further, the cloaking-incentives on SEO and SEM are different. On SEM, websites are more likely to provide illegal medicines and services. On SEO,
websites are inclined to do malicious downloading and phishing. 
Further, the strategy on cloaking detection on SEO and SEM are different. The difference is crucially decided by ranking mechanisims.
In SEO, though the page rank algorithm is not fully public, it is well known that rankings are related to content, page rank and visit traffic.
In SEM, the rank algroithm depends on real time biding system. Websites need to bid the "pay per click" price on the real time biding system. 
The one with the the higher price will be shown in higher priority. The terms with large commerical potential ask higher price. Because of the 
difference between SEM and SEO, the strategy to detect cloaking should change. The cloaking detection algorithm should change such as collecting commercial related terms, 
crawling adverstisements, checking the SEM policy(Google Ads Policy). To ensure "Pay per click" mechanisms, in other words,
websites won't pay much extra money because of cloaking detection, we introduced a new model and "click counting" mechanisms.
Traditional methods failed to do this adjust on SEM. Thus, the traditional methods are inefficient detect cloakings in SEM.
\subsection{Cloaking Types}
To understand how cloaking works in different scenarios, we will discuss the cloaking types.
In order to serve targeted users cloaking content, the scammer must use some identifiers to distinguish 
user segments. Based on these used identifiers, the cloaking techniques are classified as Repeat Cloaking, 
User Agent Cloaking, Referred Cloaking and IP Cloaking. 

In Repeat Cloaking, websites store the visit history in user-end(Cookies) or server-end (server log). Based on visit history, the website presents different information. 
According to ~\cite{wang2011cloak}, they oberved websites that only show cloaking at first time, in the hopes of making a sale, but subsequent visits are presented 
with benign page. In our observation, some repeat cloaking websites are willing to 
show the same content as the first time. 
In User Agent Cloaking, websites check the User Agent Field in the HTTP request. From the User agent field, they could find the crawlers that uses the well-known User-Agent
strings and identify crawlers.
In Referrer Cloaking, websites examamine the referer field of HTTP request. From the referer field, the website could easily find if the users clicked though search engines to
reach their websites. In this way, the cloaking websites only serves the scam page to the targeted users from search engines. 
In IP Cloaking, websites determine the visitors' identities by their IP addresses. With an accurate mapping between IP addresses
and ornizations, the websites could easily distinguish cralwers from search engines and real users. 


\subsection{Previous Work}
\subsubsection{Cloaking Detection}
The major challenge in claoking detection, is to differentiate dynamic pages
from blatantly different content.
Comparing document word by word is very expensive and slow, therefore, 
various ways to test similarity of documents are proposed. 
~\cite{henzinger2002challenges}
 considered cloaking as one of the major search engine spam
techniques.
 ~\cite{najork2005system} proposed a method of detecting cloaked pages from browsers by
 installing a toolbar. The toolbar would send the signature of user perceived
 pages to search engines.
 ~\cite{wu2006detecting} use statistics of web pages to detect cloaking,
 % common terms
~\cite{chellapilla2006improving} detected syntactic on the most popular and
 monetizable search terms. They showed that monetized search terms had a higher
 prevalence of cloaking than popular terms.
 Referrer cloaking was first studied by ~\cite{wang2006detecting}. They found
 a large number of referrer cloaking pages in their work.
 ~\cite{lin2009detection} use tag based methods,

 ~\cite{wang2011cloak}
 extended their previous efforts to examine the dynamics of cloaking over five
 months, identifying when distinct results were provided to search engine
 crawlers and browsers.
They used text-based method and tag-based method to detect Cloaking page.
 ~\cite{deng2013uncovering} use summarize previous work and compare text, tag,
 link based approaches. However, JavaScript redirects were not able be handled by their crawler.

However, non of them take into consideration the data collection part. Nor do
they take into account the efficiency of the algorithms.

In order to detect cloaking, we introduce Simhash-based Website Model (SWM) and
design and implement a much more efficient algorithm based on SWM, which has
comparable false positive rate and true positive rate. We take into account the
efficiency of the algorithm and implement a plugin which introduce negligible
overhead to user. With crowdsourcing, we can solve cloaking at scale.

%Previous work on phishing. We are solving cloaking, which has a strong
%relationship with phishing sites.

\subsubsection{Simhash}
Charikar's simhash~\cite{charikar2002similarity} has been widely used in near
duplicate detection in search engine. ~\cite{henzinger2006finding}
conducted a large-scale evaluation of simhash against Broder's shingle-based
fingerprints~\cite{broder1997syntactic} in finding near-duplicate web pages.
A great advantage of using simhash over shingles is that it
requires relatively small-sized fingerprints.
For example, ~\cite{broder1997syntactic} requires 24 bytes per fingerprint.
In comparison, ~\cite{manku2007detecting} shows that for 8B web pages, 64-bit
fingerprints suffice.

These approaches are mainly focusing on using simhash to signature websites on
the Internet and comparing them all together to identify near duplicates.
In contrast, this work leverages the fact that content from same url is supposed
to be near duplicate and employs simhash to do outlier detection.

This work adopts the same simhash algorithm and setting as 
~\cite{manku2007detecting}, but with different method of extrating text
features. ~\cite{manku2007detecting} extracts text features from website with
standard IR techniques, and weigh each feature with inverse document frequency.
This requires extra information and introduces overhead if deployed in browser at user side.
In ~\autoref{s:methodology}, we describe our approach of extracting text features.
Inspired from previous cloaking work~\autoref{wang2011cloak} which compares both
text and dom tree information, we extract dom features as well.





\section{Simhash-based Website Model}
\label{s:methodology}
The major challenge in claoking detection, is to differentiate dynamic pages
from blatantly different content.
Comparing document word by word is very expensive and slow, therefore, 
various ways to test similarity of documents are proposed. 
 ~\cite{wu2006detecting} use statistics of web pages to detect cloaking,
 ~\cite{lin2009detection} use tag based methods,
 ~\cite{deng2013uncovering} use summarize previous work and compare text, tag,
 link based approaches.

in order to detect cloaking, we design a much more efficient algorithm with
comparable false positive rate and true positive rate.

There are two challenges in cloaking detection: reveal the content and handle
dynamics of a webpage.
In order 

In order to detect cloaking,
Challenges and solutions. Basic ideas are 1, 2, 3.




\subsection{Simhash-based Website Model}
Methodology of dynamic modeling of web pages
\subsubsection{Distance Approximation}
Simhash~\cite{charikar2002similarity} is a hash function family that maps a high dimension dataset into fixed
bits and preserves the following attribute:

Suppose P and Q are probability distributions over L, 
\begin{multline}
  EMD(P, Q) \le E[d(h(P), h(Q))] \\
  \le O(\log{n}\log{\log{n}})EMD(P, Q).
\end{multline}

This equation is telling us that the hamming distance between simhash of set
\b{P} and set \b{Q} is an approximation of Earth Moving Distance(EMD) between set P
and Q. Charikar~\cite{charikar2002similarity} give the formal proof that the
hamming distance of sets represents the cosine similarity.
~\cite{manku2007detecting} implements an algorithm for creating text-based
simhash for a website. In our work, we use the same simhash algorithm,


\subsubsection{Insights On Feature Set Selection}



\subsubsection{Computation}


\subsubsection{Text Simhash and Dom Simhash}


In order to compress text information of a document, we extract the same set of
text features as ~\cite{manku2007detecting}.

Inspired from previous work, we understand that looking at only text would raise
high false positive, therefore, we take into consideration the tag.

In order to compress the structural 

for compressing
the text information. 



In order to detect cloaking, we need to capture the bahavior and similarity that
a same website maintains. That is to say, we not only need to look at the text-based simhash,
but also dom-based simhash. We implemented the text-simhash algorithm described
in ~\cite{manku2007detecting}, which extract words, bi-gram, tri-gram set
(repeated elements only recorded once) from a website and compute simhash using
simhash algorithm described in ~\cite{charikar2002similarity}.

There is no current algorithm for generating dom simhash. Therefore, we design
an algorithm to perform this task. For each dom tree, we record the node set, as
well as the child parent pair set. The node set tells us information about what
tag is present in this page, and child parent pair tells us how these tags are
organized.

% you can also use the wonderful epsfig package...
\begin{figure}[t]
  \centering
  \begin{subfigure}
    \centering
    \includegraphics[width=.5\textwidth]{fig/yahoo-text-user}
    \label{fig:yahoo-text-user}
  \end{subfigure}%
  \begin{subfigure}
    \centering
    \includegraphics[width=.5\textwidth]{fig/yahoo-dom-user}
    \label{fig:yahoo-dom-user}
  \end{subfigure}
  \caption{Yahoo simhash changes over 7x24 period Feb.1 - 7, 2015}
  \label{fig:yahoo-simhash}
\end{figure}





It is pretty straightforward from ~\autoref{fig:yahoo-simhash} that,
text simhash changes rapidly, indicating dynamic nature of this
website, and dom simhash changes relatively slow and less.

Till now, we have demonstrated the algorithm we are using to generate
text-simhash and dom-simhash out of a website. Based on our observation, the
text-simhash may change rapidly, while dom-simhash relatively remain the same.


\subsubsection{Aggregating / Clustering}
Assume we are monitoring the same website over a period of time. This website
have dynamic changes all the time. But there can be another kind of change -
whole page change. In this case, it is reasonable to first separate them apart
and look at each of them.


Hamming distance is a special case of Euclidean distance. We can take the
average to represent the center of these points.


Using the hamming distance measure on a dimension of 64-bit.


For different websites, simhash can be considered as an algorithm to map them to
a 64-bit number randomly ~\cite{manku2007detecting}. For the same website,
simhash measures the similarity between them.



\begin{figure}[t]
  \centering
  \begin{subfigure}
    \centering
    \includegraphics[width=.5\textwidth]{fig/yahoo-dom-user}
    \label{fig:yahoo-dom-user}
  \end{subfigure}%
  \begin{subfigure}
    \centering
    \includegraphics[width=.5\textwidth]{fig/yahoo-dom-google}
    \label{fig:yahoo-dom-google}
  \end{subfigure}
  \caption{Comparison of user and google seen dom simhash}
  \label{fig:yahoo-simhash}
\end{figure}




Different from ~\cite{manku2007detecting}, we not only want to know whether two pages are
duplicate, we also want to know the patterns of these simhash. In this work, we employ
hierarchical clustering to do this job.

In hierarchical clustering uses a set of dissimilarities for the n objects being clustered.
Initially, each object is assigned to its own cluster and then the algorithm
proceeds iteratively, the complete linkage method finds similar clusters.

In order to decide the number of clusters to take in hierarchical clustering
(when to stop), we use inconsistent coefficient.


L1 norm : sum of the differences in
each dimension

considerations of significance, we ask whether this is an unusual result or
whether it could have arisen merely by chance


The inconsistency coefficient characterizes each link in a cluster tree by
comparing its height with the average height of other links at the same level of
the hierarchy. The higher the value of this coefficient, the less similar the
objects connected by the link.

One way to determine the natural cluster divisions in a data set is to compare
the height of each link in a cluster tree with the heights of neighboring links
below it in the tree.

linkage metric: hammming
method: Unweighted average distance (UPGMA)
cutoff: inconsistent value less than c
pick inconsistent value now!!!!!

learning 1.1
detection 2.0


\begin{gather*} \label{npa}
  d(u,v) = \min(dist(u[i],v[j])) \\
  \text{for all points i in cluster u and j in
  cluster v. }
\end{gather*}
This~\autoref{npa} is known as the Nearest Point Algorithm.

Single-linkage clustering is one of several methods of agglomerative
hierarchical clustering. In the beginning of the process, each element is in a
cluster of its own. The clusters are then sequentially combined into larger
clusters, until all elements end up being in the same cluster. The stop
criterion is the distance one.


\subsection{Data collection and Groundtruth}

1. Collect terms
2. Query search results.
3. Crawl data and get ground truth (how
many).
4. Train model and select parameters use 5-fold stratified cross validation
~\cite{scikit-learn}.

\subsection{Cloaking Detection}



Classification hinges on having access to a robust set of features derived from
URLs to discern between spam and non-spam. Previous work has shown that lexical
properties of URLs, page content, and hosting properties of domains are all
effective routes for classification [15], [16], [22]–[24]. We expand upon these
ideas, adding our own sources of features collected by one of three components:
a web browser, a DNS resolver, and IP address analysis. A comprehensive list of
features and the component that collects them can be found in Table 1. A single
monitor oversees multiple copies of each component to aggregate results and
restart failed processes. In turn, the monitor and feature collection components
are bundled into a crawling instance and replicated in the cloud


%\subsection{Model Selection}
%
%
%\begin{table*}[!th]                                                     
%  \centering                                                            
%  \scriptsize                                                           
%  \begin{tabular}{lllllllllll}
  \toprule
  & \multicolumn{2}{c}{\textbf{Normal}}
  & \multicolumn{2}{c}{\textbf{Lognormal}}
  & \multicolumn{2}{c}{\textbf{Exponential}}
  & \multicolumn{2}{c}{\textbf{Gamma}}
  & \multicolumn{2}{c}{\textbf{Logistic}}\\

  \textbf{Website(Hash Type)\textbackslash Model}
  & AD-value
  & P-value
  & AD-value
  & P-value
  & AD-value
  & P-value
  & AD-value
  & P-value
  & AD-value
  & P-value \\
  \midrule
  digg.com T & 0.617 &  0.100 & 0.481 &  0.219 &
  14.851 & < 0.003 & 0.538 &  0.186 & 0.531 &  0.131\\ 
  digg.com T & 0.227 &  0.806 & 0.179 &  0.914 &
  19.690 &  < 0.003 & 0.198 &  > 0.250 & 0.250 & >0.250\\
  yahoo.com T & 0.192 &  0.893 & 0.263 &  0.692 &
  35.828 & <0.003 &   0.231 & >0.250 & 0.222 & >0.250\\
  amazon.com T & 0.720 &  0.058 & 0.323 &  0.520 & 
  27.754 & <0.003 &  0.436 & >0.250 & 0.642 &  0.058\\
  reddit.com T & 0.373 &  0.411 & 0.331  & 0.509 & 
  35.063 & <0.003 & 0.340 & >0.250 & 0.361 & >0.250\\
  yacombinator.com T & 0.473 &  0.237 & 0.516 &  0.186 &
  37.551 & <0.003 & 0.519 &  0.204 & 0.583 &  0.089\\

  digg.com D & 0.319 &  0.372 & 0.348 &  0.305 &
  1.491 &  0.021 &  0.402 & >0.250 & 0.363 & >0.250\\
  yahoo.com D & 0.531 &  0.168 & 0.392 &  0.366 &
  18.837 & <0.003 & 0.441 & >0.250 & 0.584  & 0.088\\
  amazon.com D & 1.519 & <0.005 & 0.916 &  0.019 &
  22.083 & <0.003 & 1.052 &  0.009 & 0.548 &  0.114\\
  amazon.com D & 0.483 & 0.117 &  0.504 & 0.104 &
  1.741 & 0.010 & 0.601 & 0.128 & 0.523 & 0.115\\

\end{tabular}

                                     
%  \caption{Model statistics for selected websites}
%  \label{tbl:para-select}                                         
%\end{table*}                                                            
%
%
%This table ~\autoref{tbl:para-select} shows the Anderson-Darling (AD) value and P-value for each model.
%A common rule used in model selection is pick the model which has the smallest
%value with P-value greater than 5\%. Each row in the table represents one
%website. From the statistics of these websites, we choose normal distribution
%for text simhash and Lognomal distribution for dom simhash.
%
%In the simhash based cloaking detection model, input from the user is simply simhash. How to compare against the simhashs that is already collected?
%
%One simple way is to compute the average distance from this simhash to all the observed simhashs. The next step is then to tell whether this distance is reasonable. 
%
%The text distribution follows lognormal distribution.
%After mannual check of those results.
%
%
%
%
%You have to run {\tt latex} once to prepare your references for
%munging.  Then run {\tt bibtex} to build your bibliography metadata.
%Then run {\tt latex} twice to ensure all references have been resolved.
%If your source file is called {\tt usenixTemplate.tex} and your {\tt
%bibtex} file is called {\tt usenixTemplate.bib}, here's what you do:
%{\tt \small
%  \begin{verbatim}
%  latex usenixTemplate
%  bibtex usenixTemplate
%  latex usenixTemplate
%  latex usenixTemplate
%  \end{verbatim}
%}
%



\section{Evaluation}
\label{s:evaluation}


%1. Collect terms
%2. Query search results.
%3. Crawl data and get ground truth (how
%many).
%4. Train model and select parameters use 5-fold stratified cross validation
%~\cite{scikit-learn}.

In the above section ~\autoref{s:methodology}, we propose the SWM and explains
how it can be used to do cloaking detection (outlier detection). There are three
parameters to be learned, the upper bound of inconsistent coefficient in
clustering phase $T_{learn}$, the lower bound of inconsistent coefficient in the detection
phase $T_{detect}$, the fix parameter minimum radius $R_{detect}$. 
In order to measure cloaking in both SEO and SEM, we collects four candidate dataset
~\autoref{ss:dataset}. In this section,
we first describe the groundtruth obtained from $D_{hot, search}$ and $D_{spam,
search}$, then use it to train and test the performance of the proposed model.

\subsection{Groundtruth}

%We randomly sample 600 websites from the dataset, for 10 times. This results in
%5726 websites. We manually label them and \XXX{cloaking}, \XXX{not}, percentage
%for each is.
Similar to ~\cite{lin2009detection}, we start by remove duplicates (same simhash
from user view and Google view) from $D_{hot, search}$ and $D_{spam, search}$,
because these are not helpful for the algorithm training (designed to handle
dynamics of websites, non-changing websites are handled by default). Then we
manually label websites from $D_{hot, search}$ and $D_{spam, search}$ until we
have relatively large sample size for training. By conducting this massive
labeling, we collect 1195 cloaking examples. In terms of normal websites, we randomly
select 5308 samples from non-cloaking dataset. The two parts, 6503 urls in
total, are combined as groundtruth $D_{g}$ for algorithm training and evaluation.
By applying feature extraction and simhash computation described in
~\autoref{ss:swm}, we have each url associated with its DOM simhash $S_{g, DOM}$ 
and text simhash $S_{g, text}$.

%\subsubsection{De-duplication}
%71116 urls
%
%62042 websites
%
%This results in \XXX{Some} links. Then we compare the text simhash and dom
%simhash, remove those which are exactly duplicate of one of the simhash observed
%by Google. After this step, we have \XXX{N} url left.
%
%For advertisements, after deduplication, there are 997 (score 60) urls remained.
%
%For search results, after deduplication 37155 urls, 35444 websites remained.



%We remove the failure websites and this results in 
%113242 urls, exact match, parameter different are counted.
%98390 sites, parameters ignored. Later we will use the latter parameter because
%it makes more sense.

%Step 1: Filter
%
%In order to get groundtruth, we follow a similar process employed in
%~\cite{lin2009detection}, we first filter the results and get rid of the highly reputated ones. We write a
%script to query the WOT API, and remove websites with combined score 80
%(which is a pretty high score) and the results are \XXX{N} urls after that.
%
%for advertisements, after filtering, there are 2279 (score 60) urls
%remained.\XXX{Problematic because I haven't merged them}
%
%for search results, after filtering, there are 90120 (60) urls remained.
%\XXX{Problematic because I haven't merged them}
%

%\subsubsection{Random Sample and Labeling}
%Then we randomly select 1000 urls from the dataset, and label them, after
%labeling, we found \XXX{N} cloaking sites and \XXX{M} dynamic websites.
%These are the groundtruth we used to label our data.


\subsection{Detection and Evaluation}
Regarding parameters to learn, $T_{learn}$ and $T_{detect}$ are parameters to
handle page dynamics, and $R_{detect}$ is a fix parameter to make system 
robust to consistent difference between spider and user copies, which doesnot
contribute to website modeling. Therefore, 
we first select optimal $T_{learn}$ and $T_{detect}$, and then present
system performance for different settings of $R_{detect}$.

\subsubsection{Selection of $T_{learn}$ and $T_{detect}$}
\label{sss:threshold}
Because $R_{detect}$ is a parameter to allow the system to handle consistent
difference between spider and user copies, therefore, we first set detect
$R_{detect}$ to be zero, and do five-fold stratified cross validation~\cite{scikit-learn}.
on with groundtruth $D_{g}$. In the learning phase,
our objective function is to first minimize the total number of errors in
classification $E = FP + FN, FP = \text{false positive}, FN = \text{false
negative}$, and
if $E$ is the same, minimize $d = T_{detect} - T_{learn}$.
This is reasonable because $d$ is the area that we cannot reject or accept. The smaller
the area, the more compact the learned model.
%The two objective
%function are widely used metrics in machine
%learning parameter selection \XXX{cite}.

By applying five-fold stratified cross validation on $S_{g, DOM}$ and $S_{g, text}$ and the described
objective function for optimal parameter selection, DOM simhash returned 
$T_{detect, DOM} = 1.8$ and $T_{learn, DOM} = 0.7$, and text simhash yields
$T_{detect, text} = 2.1$ and $T_{learn, text} = 0.7$.
% this is meaningless, i think
%
% majority is false positive.
% which yields to distance 1.1 at optimal (1318.4 learn error, 329 detect error), 

\subsubsection{Radius Selection $R_{detect}$}
Similarly, $R_{detect, text}$ and $R_{detect, DOM}$ are decided separately. In this section, we conduct
three experiments: cloaking detection using (1) DOM simhash (2) text simhash (3)
intersection DOM simhash and text simhash result.
Again, we use five-fold stratified cross validation and same objective function as in
~\autoref{sss:threshold} to learn and test $S_{g, DOM}$ and
$S_{g, text}$. The
optimal parameter for DOM simhash is $R_{detect, DOM} = 17$, text simhash is
$R_{detect, text} = 16$. ~\autoref{fig:roc} gives an illustration on
how False Positive Rate (FPR) and True Positive Rate (TPR) changes as
threshold for DOM and text changes. 
%If we consider significant difference in both text and dom as
%cloaking, the result is $R_{detect, dom} = 13$,
%$R_{detect, text} = 17$, and we get $FPR = 0.1\%, TPR = 94\%$.

Next, in order to show the combined result for different settings of $R_{detect,
dom}$, $R_{detect, text}$, we set $R_{detect, text}$ around its optimal value
and change $R_{detect, dom}$ as shown in ~\autoref{fig:roc}. It is obvious that
combining both DOM and text features improved the performance. 
The learned parameters are used in ~\autoref{s:measurement} to detect cloaking
on the four dataset, and since we want to capture as much cloaking as possible
for incentive study, we choose $R_{detect, text} = 15$ and $R_{detect, DOM} =
13$, which corresponds to 0.3\% FPR and 97.1\% TPR.

% you can also use the wonderful epsfig package...
\begin{figure}[t]
  \centering
  \includegraphics[width=.5\textwidth]{fig/roc}
  \caption{ROC for DOM, TEXT, DOM \& TEXT}
  \label{fig:roc}
\end{figure}



\section{Measurement}
\label{s:measurement}

\subsection{Dataset}
Get detected results from 60K (scale of previous research) data and analyze. 

\subsection{Cloaking in SEO}

SEO: How severe is cloaking? How many categories and percentages of various
cloaking? For each type of cloaking, what are their incentives?

\XXX{Plot a log scale pie or histogram for categories in SEO}


\subsection{Cloaking in SEM}

How severe is cloaking? How many categories and percentages of various cloaking?
For each type of cloaking, what are their incentives?

\XXX{Plot a log scale pie or histogram for categories in SEM}



\section{Discussion}
\label{s:discussion}

In this section, we first compare our work with previous cloaking detection
works, then discuss two kinds of deployment of simhash-based cloaking
detection system: server-based and crowdsoucing-based. We analyze the attack models
and robustness of crowdsourcing model.

\subsection{Efficiency Comparison}
\label{ss:efficiency}
Previous cloaking detection approaches use different feature sets from a website,
and the latest work ~\cite{wang2011cloak} use text features, dom features and
search snippet to detect cloaking. Comparing with their model, we achieve
similar precision and recall (transformed from FPR and TPR). However, a great
advantage of our work over past approaches is that, our algorithm is
efficient and clean: requires only single pass of website to get fuzzy signature and then
compare fuzzy signatures instead of the original document. This feature 
not only saves time and make cloaking detection scalable, 
but also enables data collection to be deployed at user side for crowdsourcing.

~\autoref{comparison} compares past cloaking detection approaches. Our approach
use the content that contains most information (entropy), and is more efficient
compared to past approaches. Regarding text simhash and DOM simhash generation,
we implement a browser plugin and it introduces little overhead to browsing
session. Besides, in order to further reduce overhead to browser, simhash
computation can be triggered under some circumstances. For example, compute
simhash only when advertisements are visited (url contains specific string), and
set computation probability to $p = \frac{1}{100}$ to further reduce overhead.

%Compared with past approaches, we achiev
%While we can achieve similar false positve rate and true positive rate compared
%to past approaches, we argue that, our approach is much more efficient than past
%approaches and is light-weight enough to be deployed on user browser.
%\XXX{Table} is a comparison of time complexity and number of rounds that the
%documents need to be processed (each time they get smaller amount of document,
%though). Let $N$ denote the number of total urls collected, $M$ denoete the
%number of cloaking websites. The time complexity is \XXX{Plot}.

\begin{table*}[t]
  \centering
  \begin{center}
    \begin{tabularx}{1\textwidth}{c|c|c|c|c|c}
    \toprule
    \multicolumn{6}{c} {$W$: number of words, $T$: number of tags,
    $L$: number of links} \\
    \midrule
      Methods & Features & Complexity & Performance &
      Efficiency & Deployment \\
      \hline
      Najork ~\cite{najork2005system} & terms, links, tags 
      & $O(W^2 + L^2 + T^2)$ & medium & low & Server \\
      Term \& Link Diff ~\cite{wu2005cloaking}       & terms and links & $O(W^2 + L^2)$
      & low & low & Server  \\
      Wu \& Davison ~\cite{wu2006detecting} & terms, links, tags & $O(W^2 + L^2 + T^2)$ & 
      high & low &  Server \\
      CloakingScore ~\cite{chellapilla2006improving} & terms &   $O(W^2 +
      L^2 + T^2)$ & high & low & Server  \\
      Click-Through ~\cite{wang2006detecting} & click-through  & 
      $O(W^2 + L^2 + T^2)$ & high & low & Server  \\
      TagDiff ~\cite{lin2009detection} & tags  &    $O(T^2)$  & medium &
      high &  Server  \\
      Cloaker and Dagger ~\cite{wang2011cloak} & terms, tags  &  $O(W^2 +
      T^2)$ & high & medium & Server \\
      Hybrid Detection ~\cite{deng2013uncovering} & terms, links, tags &
      $O(W^2 + T^2)$ & high & low & Server \\
      SWM & terms, tags & $O(W + T)$ & high & high & Server and User \\
      \bottomrule
    \end{tabularx}
  \end{center}
  \caption{Comparison of cloaking detection methods}
  \label{comparison}
\end{table*}






\subsection{Server-based Deployment}
In server-based cloaking detection system,
we first collect targeted search terms includes commecial terms, cloaking oriented terms and hot trend words.
Using these search terms, the system retrieves the search results from the search engines for seven times.
First, the system disguises as normal user by using normal user agent. Next, the
system disguises as Google crawler by using the Google Agent and visit landing
pages obtained from the first visit. With this data that
is crawled from Google view, the system uses the simhash method to model the websites.  
The system compares simhash value from user view and
website model learned from Google view. From the comparsion result, the system judges if the website is cloaking
or not. 
Server based cloaking detection system has pros and cons. The deployment of server based cloaking dectetion system
is pratical and could be deployed
easily. The tradeoff of easy deployment is inefficient in IP cloaking detection. Usually, the servers IP addresses
are in a range,  scammers could find this range and serve benign content to crawlers. One solution is buying numerous IP addresses
from ISP providers and distributing IP addresses as similar to real user distribution. This increases cloaking detection cost. In
addition, distributing IP addresses as users is hard. Further, server-based cloaking detection system is infeasiable
to detect cloaking in search engine marketing(SEM). As we mentioned, using the crawlers to visit websites in SEM field
increases the advertisement cost of websites. Moreover, different websites has
different changing periods. For example, {\it www.yahoo.com} 
updates fast, but {\it www.apple.com} is relatively slow. Identifying different 
crawling periods for different websites is difficult. 

\subsection{Crowdsourcing-based Deployment}
Crowdsourcing cloaking detection system includes user-side and server-side component. In user side, users needs to install the cloaking
detection extension in their browers. While users click search results and view
websites, the extension computes the simhash value based on 
website content and layouts. After calculation, extension packs URL and simhash
value and sends to server. Server passively receives $(URL, simhash value)$
pairs from users. Server also utilizes crawlers to extract content several times
from this URL from search engine view (crawler uses Google bot agents).
Server uses these extracted content to model the website. Through comparing
$(URL, simhash value)$ pair from users and crawler view,
server could decide if the website is cloaking or not. As mentioned in
~\autoref{s:measurement}, incentives behind cloaking includes domain parking
~\cite{vissers2015parking} (suspicious) and
malware distribution. Based on cloaking
detection result, server categorizes cloaking and update blacklists in browsers' extensions.
Updating blacklists and warning users phishing and
malware downloading is users incentive to install our extension.

~\autoref{fig:workflow} explains the proposed crowdsourcing-based cloaking detection framework. 
Next, We discuss the pros and cons for this approach. The first advantage is privacy. Instead of solicting
website content from users, the system solicited a 64 bits simhash value from users. 
From this 64 bits value, system couldn't do reverse engineering to
get original content. In addition, the system could intergrate with 
RAPPOR~\cite{erlingsson2014rappor}, which protect the privacy of URL.
Because the workflow is similar to safe browsing API ~\cite{rajab2013camp} , we argue that this can be easily
extended in current framework, the only different is that (URL, simhash value)
pair is processed through RAPPOR to achieve anonymity.

In addition, the crowdsouring deployment introduces low traffic. Browsers only send a 64 bits value for
each URL. This won't jam traffic. Further, the crowdsoucing deployment 
wouldn't affect the model of SEM. Each click on advertisements of search engines is
from real users instead of crawlers. Advertisers don't need to pay extra money for cloaking detection. 


%Employ RAPPOR~\cite{erlingsson2014rappor} to provide user privacy gurantee.
%
%pros: 	1.privacy 2.Low traffic. 3.SEM 4.Distributed computation 
%5. Remove the need to do redirect cloaking detection, leveraging the feature
%that the end goal of attackers is to reach user
%6. could decide crawl period passively based on user clicks, data received are
%based on real user’s clicks, say, website traffic
%
%cons: user incentives. 
%Solution: Plugin to detect suspicious websites. API


\begin{figure}[t]
  \centering
  \includegraphics[width=.45\textwidth]{fig/crowdsourcing-cloaking-detection-system}
  \caption{Workflow of crowdsource cloaking detection sysetem}
  \label{fig:workflow}
\end{figure}

%The workflow ~\autoref{fig:workflow} is to collect page contents simhash on the user side, and compare
%them to simhash of the same link from ad serving company to find cloaking. When
%the differences of the simhashes are significantly large, the page is marked
%cloaking. We generate two simhash for page content and structure respectively.
%Intuition behind this is, simhash difference between different sites are larger
%than different visits of the same site. We build a two-phase system to detect
%cloaking: cluster learning phase, and cloaking detection phase. In the cluster
%learning phase, an ad company visit urls and generate simhash from its content
%with its owned IP, and learn pattern and distribution of the simhashes, i.e.
%simhash-based website model. In the cloaking detection phase, the ad company
%collects simhash from its users. Compare them with learned patterns, return
%cloaking score or mismatch
%


\section{Conclusion}
\label{s:conclusion}

Cloaking is an ongoing war between inspectors and cloakers. 
In the field of SEO and SEM, cloakers have different incentives for cloaking,
i.e. monetizing traffic versus providing illegal service.
The two challenges in cloaking deteciton enables cloaking to be popular in the
wild: revealing cloaked content and diffrentiate dynamics from cloaking.
This work proposes Simhash-based Website Model to address these challenges and
achieves 97\% true positive rate at a false positive rate of 0.3\%.  

A great advantage of this
solution compared to past approaches is efficiency and ability to be deployed as
user plugin. By crowdsourcing user views, both challenges are addressed
elegantly. While we are not able to deploy
the crowdsourcing model, we implement the data collection as browser plugin and
encourage search engine or other inspectors to adopt this solution.

%Previous work on simhash focuses on using simhash to find near duplicates, in
%contrast, this work leverage the assuption that content from same website are
%likely to be near duplicates and use simhash to learn clusters and model website
%dynamics.



%design is suitabl
%
%Current solution, multiple pass of data, doesn’t work, and hard to
%handle various types of cloaking.
%Our model works because …
%



%
%\section{Acknowledgments}
%
%A polite author always includes acknowledgments.  Thank everyone,
%especially those who funded the work. 
%
%\section{Availability}
%
%It's great when this section says that MyWonderfulApp is free software, 
%available via anonymous FTP from
%
%\begin{center}
%{\tt ftp.site.dom/pub/myname/Wonderful}\\
%\end{center}
%
%Also, it's even greater when you can write that information is also 
%available on the Wonderful homepage at 
%
%\begin{center}
%{\tt http://www.site.dom/\~{}myname/SWIG}
%\end{center}
%
%Now we get serious and fill in those references.  Remember you will
%have to run latex twice on the document in order to resolve those
%cite tags you met earlier.  This is where they get resolved.
%We've preserved some real ones in addition to the template-speak.
%After the bibliography you are DONE.

{\footnotesize \bibliographystyle{acm}
%\bibliography{../common/bibliography}}
\bibliography{ruian,weiren}}
\clearpage
% \section*{Appendix}

% \begin{appendix}
% \section{
% \subsection{Deploying Solutions into Android Ecosystem} 
% \label{app:summary-table-wrapper}

%\begin{appendix*}[!h]
%\begin{figure*}[t]
%  \centering
%  \subfigure[Search on Google: instant loan quote]{
%    \centering
%  \includegraphics[width=.4\textwidth]{fig/loan-quote-search}
%  %\subcaption{Search on Google: Essary Writing}
%  \label{seo:search}}
%\subfigure[User is presented with loan quote service]{
%    \centering
%  \includegraphics[width=.4\textwidth]{fig/loan-quote-user-visit}
%  %\subcaption{User is presented with dishonest service.}
%  \label{seo:user}}
%\subfigure[Spider is presented with SEO content]{
%  \centering
%  \includegraphics[width=.4\textwidth]{fig/loan-quote-google-visit}
%  %\subcaption{Spider is presented with hotel page.}
%  \label{seo:google}}
%\subfigure[Victim site]{
%  \centering
%  \includegraphics[width=.4\textwidth]{fig/loan-quote-victim}
%  %\subcaption{Spider is presented with hotel page.}
%  \label{seo:victim}}
%\caption{Cloaking for Phishing Example on Google Search }
%\end{figure*}
%


\begin{figure*}[t]
  \centering
  \subfigure[Search on Google: Essary Writing]{
  \includegraphics[width=.6\textwidth]{fig/search}
  %\subcaption{Search on Google: Essary Writing}
  \label{sem:search}
}

\subfigure[User is presented with dishonest service]{
  \includegraphics[width=.4\textwidth]{fig/ghostwrite}
  %\subcaption{User is presented with dishonest service.}
  \label{sem:user}
}
\subfigure[Spider is presented with hotel]{
  \includegraphics[width=.4\textwidth]{fig/hotel}
  %\subcaption{Spider is presented with hotel page.}
  \label{sem:google}
}
\caption{Claoking Example on Google Search Advertisement}
\end{figure*}
%\end{appendix*}

% \end{appendix}​}<++>

%\theendnotes

\end{document}



