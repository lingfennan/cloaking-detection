% TEMPLATE for Usenix papers, specifically to meet requirements of
%  USENIX '05
% originally a template for producing IEEE-format articles using LaTeX.
%   written by Matthew Ward, CS Department, Worcester Polytechnic Institute.
% adapted by David Beazley for his excellent SWIG paper in Proceedings,
%   Tcl 96
% turned into a smartass generic template by De Clarke, with thanks to
%   both the above pioneers
% use at your own risk.  Complaints to /dev/null.
% make it two column with no page numbering, default is 10 point

% Munged by Fred Douglis <douglis@research.att.com> 10/97 to separate
% the .sty file from the LaTeX source template, so that people can
% more easily include the .sty file into an existing document.  Also
% changed to more closely follow the style guidelines as represented
% by the Word sample file. 

% Note that since 2010, USENIX does not require endnotes. If you want
% foot of page notes, don't include the endnotes package in the 
% usepackage command, below.

% This version uses the latex2e styles, not the very ancient 2.09 stuff.
\documentclass[letterpaper,twocolumn,10pt]{article}
\usepackage{usenix,epsfig,endnotes}
\usepackage{graphicx,calc,subfigure,caption,float}
\usepackage[breaklinks,colorlinks]{hyperref}
\usepackage{endnotes,microtype,xspace,fancyvrb,multirow}
\usepackage{array,underscore,relsize}

\input{cmds}

\begin{document}

%don't want date printed
\date{}

%make title bold and 14 pt font (Latex default is non-bold, 16 pt)
\title{\Large \bf Simhash-based Website Model: Towards Real-Time Cloaking Detetion}

%for single author (just remove % characters)
\author{
{\rm Your N.\ Here}\\
Your Institution
\and
{\rm Second Name}\\
Second Institution
% copy the following lines to add more authors
% \and
% {\rm Name}\\
%Name Institution
} % end author

\maketitle

% Use the following at camera-ready time to suppress page numbers.
% Comment it out when you first submit the paper for review.
\thispagestyle{empty}


\subsection*{Abstract}
Cloaking, used by spammers for the purpose of increasing the visiting rates of
their website, as well as to circumvent censorship, has been a challenging
spamming technique to search engines. Recently, there is a rising trend of
employing
cloaking in advertisement spam ~\cite{li2012knowing}.  The motivation of
cloaking 
is to hide the true nature of a website by delivering different semantic
content users versus censors.
Cloaking is popular due to low cost to setup and lack of efficient technique
to detect.

There are two major challenges in cloaking detection (A) revealing the
uncloaked content 
(B) differentiating cloaking from naturally dynamic changes of the same page.
This work employs
simhash-based website model to detect cloaking. The key idea to address
challenge 
(A) is to crowd-source data collection to user. However, collect such
information 
introduces large volume of traffic and violates user privacy. 
This work proposes simhash-based website model to address challenge (B)
with minimum amount of data and guarantee of user privacy.

In order to show the effectiveness of simhash-based website model, we build
a prototype and results 
show that we can achieve more than 95\%? true positive rate with 5\%? false
positive rate.





\section{Introduction}
\label{s:intro}

Cloaking, used by spammers for the purpose of increasing the visiting rates of
their website, as well as to circumvent censorship, has been a challenging
spamming technique to search engines. Recently, there is a rising trend of
employing cloaking in advertisement spam~\cite{li2012knowing}.  The motivation
of cloaking is to hide the true nature of a website by delivering different
semantic content users versus censors.

Spammers use user agent, referer, IP etc based cloaking techniques to
differentiate censors and normal users.

Cloaking is popular due to the low cost to setup and lack of efficient technique
to detect. Spammers use user agent, referer, IP etc based cloaking techniques to
differentiate censors and normal users.

There are two major challenges in cloaking detection (A) revealing the uncloaked
content (B) differentiating cloaking from naturally dynamic changes of the same
page. Existing approaches pretend to be the user, e.g. use proxies, set referer,
to reveal the uncloaked content, but spammers could detect and block them. An
intuitive way is to collect data from user directly. However, it is hard to
collect sufficient data for comparison while protecting user privacy.


The workflow is to collect page contents simhash on the user side, and compare
them to simhash of the same link from ad serving company to find cloaking. When
the differences of the simhashes are significantly large, the page is marked
cloaking. We generate two simhash for page content and structure respectively.
Intuition behind this is, simhash difference between different sites are larger
than different visits of the same site. We build a two-phase system to detect
cloaking: cluster learning phase, and cloaking detection phase. In the cluster
learning phase, an ad company visit urls and generate simhash from its content
with its owned IP, and learn pattern and distribution of the simhashes, i.e.
simhash-based website model. In the cloaking detection phase, the ad company
collects simhash from its users. Compare them with learned patterns, return
cloaking score or mismatch. Results on have shown data set show that we can
achieve more than 95\% True positive rate with 5\% false positive rate for page
content simhash and 100\% true positive rate with 2\% false positive rate for
structure simhash. If we define cloaking as mismatch in both page content
simhash and structure simhash (intersection of mismatches in both case), we can
achieve more than 99\% true positive rate with 1\% false positive rate.


This paper makes the following contributions:
(1) Introduce simhash-based website model to represent the content and layout
dynamics of a page.
(2) Propose the idea and framework to detect cloaking and protect user through crowdsourcing,
with low traffic overhead and privacy guarantee for user.
(3) Use of simhash-based website model, in cloaking detection, with better FPR
and TPR.
(4) Detect cloaking in both SEO and SEM , showing the seriousness of cloaking
problem in SEO and SEM.
(5) Examine the cloaking strategies employed by attackers.


The remainder of this paper is structured as follows. Section
~\autoref{s:related-work} provides a
technical background on cloaking and simhash, and related work in cloaking
detection. Section ~\autoref{s:swm} introduces the simhash-based website model.
Followed by a description of cloaking detection framework design in 
Section ~\autoref{s:framework}. Section ~\autoref{s:implementation} describes the 
prototype that we implemented and Section ~\autoref{s:experiment} shows our results,
followed in  Section ~\autoref{s:discussion} by a discussion of attack
model and defenses.



More fascinating text. Features\endnote{Remember to use endnotes, not
footnotes!} galore, plethora of promises.\\





\section{Related Work}

Show the story here please!


\section{Simhash-based Website Model}
\label{s:swm}
\subsection{Distance Approximation}
Simhash~\cite{charikar2002similarity} is a hash function family that maps a high dimension dataset into fixed
bits and preserves the following attribute:

Suppose P and Q are probability distributions over L, 
\begin{multline}
  EMD(P, Q) \le E[d(h(P), h(Q))] \\
  \le O(\log{n}\log{\log{n}})EMD(P, Q).
\end{multline}

This equation is telling us that the hamming distance between simhash of set
\b{P} and set \b{Q} is an approximation of Earth Moving Distance(EMD) between set P
and Q. Charikar~\cite{charikar2002similarity} give the formal proof that the
hamming distance of sets represents the cosine similarity.
~\cite{manku2007detecting} implements an algorithm for creating text-based
simhash for a website.


\subsection{Computation}
In order to detect cloaking, we need to capture the bahavior and similarity that
a same website maintains. That is to say, we not only need to look at the text-based simhash,
but also dom-based simhash. We implemented the text-simhash algorithm described
in ~\cite{manku2007detecting}, which extract words, bi-gram, tri-gram set
(repeated elements only recorded once) from a website and compute simhash using
simhash algorithm described in ~\cite{charikar2002similarity}.

There is no current algorithm for generating dom simhash. Therefore, we design
an algorithm to perform this task. For each dom tree, we record the node set, as
well as the child parent pair set. The node set tells us information about what
tag is present in this page, and child parent pair tells us how these tags are
organized.

% you can also use the wonderful epsfig package...
\begin{figure}[t]
  \centering
  \begin{subfigure}
    \centering
    \includegraphics[width=.5\textwidth]{fig/yahoo-text-user}
    \label{fig:yahoo-text-user}
  \end{subfigure}%
  \begin{subfigure}
    \centering
    \includegraphics[width=.5\textwidth]{fig/yahoo-dom-user}
    \label{fig:yahoo-dom-user}
  \end{subfigure}
  \caption{Yahoo simhash changes over 7x24 period Feb.1 - 7, 2015}
  \label{fig:yahoo-simhash}
\end{figure}





It is pretty straightforward from ~\autoref{fig:yahoo-simhash} that,
text simhash changes rapidly, indicating dynamic nature of this
website, and dom simhash changes relatively slow and less.

Till now, we have demonstrated the algorithm we are using to generate
text-simhash and dom-simhash out of a website. Based on our observation, the
text-simhash may change rapidly, while dom-simhash relatively remain the same.


\subsection{Aggregating / Clustering}
Assume we are monitoring the same website over a period of time. This website
have dynamic changes all the time. But there can be another kind of change -
whole page change. In this case, it is reasonable to first separate them apart
and look at each of them.


Hamming distance is a special case of Euclidean distance. We can take the
average to represent the center of these points.


Using the hamming distance measure on a dimension of 64-bit.


For different websites, simhash can be considered as an algorithm to map them to
a 64-bit number randomly ~\cite{manku2007detecting}. For the same website,
simhash measures the similarity between them.



\begin{figure}[t]
  \centering
  \begin{subfigure}
    \centering
    \includegraphics[width=.5\textwidth]{fig/yahoo-dom-user}
    \label{fig:yahoo-dom-user}
  \end{subfigure}%
  \begin{subfigure}
    \centering
    \includegraphics[width=.5\textwidth]{fig/yahoo-dom-google}
    \label{fig:yahoo-dom-google}
  \end{subfigure}
  \caption{Comparison of user and google seen dom simhash}
  \label{fig:yahoo-simhash}
\end{figure}




Different from ~\cite{manku2007detecting}, we not only want to know whether two pages are
duplicate, we also want to know the patterns of these simhash. In this work, we employ
hierarchical clustering to do this job.

In hierarchical clustering uses a set of dissimilarities for the n objects being clustered.
Initially, each object is assigned to its own cluster and then the algorithm
proceeds iteratively, the complete linkage method finds similar clusters.

In order to decide the number of clusters to take in hierarchical clustering
(when to stop), we use inconsistent coefficient.


L1 norm : sum of the differences in
each dimension

considerations of significance, we ask whether this is an unusual result or
whether it could have arisen merely by chance


The inconsistency coefficient characterizes each link in a cluster tree by
comparing its height with the average height of other links at the same level of
the hierarchy. The higher the value of this coefficient, the less similar the
objects connected by the link.

One way to determine the natural cluster divisions in a data set is to compare
the height of each link in a cluster tree with the heights of neighboring links
below it in the tree.

linkage metric: hammming
method: Unweighted average distance (UPGMA)
cutoff: inconsistent value less than c
pick inconsistent value now!!!!!

learning 1.1
detection 2.0


\begin{gather*} \label{npa}
  d(u,v) = \min(dist(u[i],v[j])) \\
  \text{for all points i in cluster u and j in
  cluster v. }
\end{gather*}
This~\autoref{npa} is known as the Nearest Point Algorithm.

Single-linkage clustering is one of several methods of agglomerative
hierarchical clustering. In the beginning of the process, each element is in a
cluster of its own. The clusters are then sequentially combined into larger
clusters, until all elements end up being in the same cluster. The stop
criterion is the distance one.




It can get tricky typesetting Tcl and C code in LaTeX because they share
a lot of mystical feelings about certain magic characters.  You
will have to do a lot of escaping to typeset curly braces and percent
signs, for example, like this:
``The {\tt \%module} directive
sets the name of the initialization function.  This is optional, but is
recommended if building a Tcl 7.5 module.
Everything inside the {\tt \%\{, \%\}}
block is copied directly into the output. allowing the inclusion of
header files and additional C code." \\

Sometimes you want to really call attention to a piece of text.  You
can center it in the column like this:
\begin{center}
  {\tt \_1008e614\_Vector\_p}
\end{center}
and people will really notice it.\\

\noindent
The noindent at the start of this paragraph makes it clear that it's
a continuation of the preceding text, not a new para in its own right.


Now this is an ingenious way to get a forced space.
{\tt Real~$*$} and {\tt double~$*$} are equivalent. 

Now here is another way to call attention to a line of code, but instead
of centering it, we noindent and bold it.\\

\noindent
{\bf \tt size\_t : fread ptr size nobj stream } \\

And here we have made an indented para like a definition tag (dt)
in HTML.  You don't need a surrounding list macro pair.
\begin{itemize}
  \item[]  {\tt fread} reads from {\tt stream} into the array {\tt ptr} at
    most {\tt nobj} objects of size {\tt size}.   {\tt fread} returns
    the number of objects read. 
\end{itemize}
This concludes the definitions tag.

\subsection{Model Selection}


\begin{table*}[!th]                                                     
  \centering                                                            
  \scriptsize                                                           
  \begin{tabular}{lllllllllll}
  \toprule
  & \multicolumn{2}{c}{\textbf{Normal}}
  & \multicolumn{2}{c}{\textbf{Lognormal}}
  & \multicolumn{2}{c}{\textbf{Exponential}}
  & \multicolumn{2}{c}{\textbf{Gamma}}
  & \multicolumn{2}{c}{\textbf{Logistic}}\\

  \textbf{Website(Hash Type)\textbackslash Model}
  & AD-value
  & P-value
  & AD-value
  & P-value
  & AD-value
  & P-value
  & AD-value
  & P-value
  & AD-value
  & P-value \\
  \midrule
  digg.com T & 0.617 &  0.100 & 0.481 &  0.219 &
  14.851 & < 0.003 & 0.538 &  0.186 & 0.531 &  0.131\\ 
  digg.com T & 0.227 &  0.806 & 0.179 &  0.914 &
  19.690 &  < 0.003 & 0.198 &  > 0.250 & 0.250 & >0.250\\
  yahoo.com T & 0.192 &  0.893 & 0.263 &  0.692 &
  35.828 & <0.003 &   0.231 & >0.250 & 0.222 & >0.250\\
  amazon.com T & 0.720 &  0.058 & 0.323 &  0.520 & 
  27.754 & <0.003 &  0.436 & >0.250 & 0.642 &  0.058\\
  reddit.com T & 0.373 &  0.411 & 0.331  & 0.509 & 
  35.063 & <0.003 & 0.340 & >0.250 & 0.361 & >0.250\\
  yacombinator.com T & 0.473 &  0.237 & 0.516 &  0.186 &
  37.551 & <0.003 & 0.519 &  0.204 & 0.583 &  0.089\\

  digg.com D & 0.319 &  0.372 & 0.348 &  0.305 &
  1.491 &  0.021 &  0.402 & >0.250 & 0.363 & >0.250\\
  yahoo.com D & 0.531 &  0.168 & 0.392 &  0.366 &
  18.837 & <0.003 & 0.441 & >0.250 & 0.584  & 0.088\\
  amazon.com D & 1.519 & <0.005 & 0.916 &  0.019 &
  22.083 & <0.003 & 1.052 &  0.009 & 0.548 &  0.114\\
  amazon.com D & 0.483 & 0.117 &  0.504 & 0.104 &
  1.741 & 0.010 & 0.601 & 0.128 & 0.523 & 0.115\\

\end{tabular}

                                     
  \caption{Model statistics for selected websites}
  \label{tbl:para-select}                                         
\end{table*}                                                            


This table ~\autoref{tbl:para-select} shows the Anderson-Darling (AD) value and P-value for each model.
A common rule used in model selection is pick the model which has the smallest
value with P-value greater than 5\%. Each row in the table represents one
website. From the statistics of these websites, we choose normal distribution
for text simhash and Lognomal distribution for dom simhash.

In the simhash based cloaking detection model, input from the user is simply simhash. How to compare against the simhashs that is already collected?

One simple way is to compute the average distance from this simhash to all the observed simhashs. The next step is then to tell whether this distance is reasonable. 

The text distribution follows lognormal distribution.
After mannual check of those results.




You have to run {\tt latex} once to prepare your references for
munging.  Then run {\tt bibtex} to build your bibliography metadata.
Then run {\tt latex} twice to ensure all references have been resolved.
If your source file is called {\tt usenixTemplate.tex} and your {\tt
bibtex} file is called {\tt usenixTemplate.bib}, here's what you do:
{\tt \small
  \begin{verbatim}
  latex usenixTemplate
  bibtex usenixTemplate
  latex usenixTemplate
  latex usenixTemplate
  \end{verbatim}
}




\section{Framework}
\label{s:framework}


\section{Implementation}
\label{s:implementation}
1. adaptive parameter. websites change dynamically. It's hard to give a contant  parameter for the dynamic changes on the website.
2. compare with other work. Result rate could be better. But dataset is different. Previous method probably will get lower deteciton rate on current data. 


\section{Experiment}
\label{s:experiment}


\section{Discussion}
\label{s:discussion}

In this section, we first compare our work with previous cloaking detection
works, then discuss two kinds of deployment of simhash-based cloaking
detection system: server-based and crowdsoucing-based. We analyze the attack models
and robustness of crowdsourcing model.

\subsection{Efficiency Comparison}
\label{ss:efficiency}
Previous cloaking detection approaches use different feature sets from a website,
and the latest work ~\cite{wang2011cloak} use text features, dom features and
search snippet to detect cloaking. Comparing with their model, we achieve
similar precision and recall (transformed from FPR and TPR). However, a great
advantage of our work over past approaches is that, our algorithm is
efficient and clean: requires only single pass of website to get fuzzy signature and then
compare fuzzy signatures instead of the original document. This feature 
not only saves time and make cloaking detection scalable, 
but also enables data collection to be deployed at user side for crowdsourcing.

~\autoref{comparison} compares past cloaking detection approaches. Our approach
use the content that contains most information (entropy), and is more efficient
compared to past approaches. Regarding text simhash and DOM simhash generation,
we implement a browser plugin and it introduces little overhead to browsing
session. Besides, in order to further reduce overhead to browser, simhash
computation can be triggered under some circumstances. For example, compute
simhash only when advertisements are visited (url contains specific string), and
set computation probability to $p = \frac{1}{100}$ to further reduce overhead.

%Compared with past approaches, we achiev
%While we can achieve similar false positve rate and true positive rate compared
%to past approaches, we argue that, our approach is much more efficient than past
%approaches and is light-weight enough to be deployed on user browser.
%\XXX{Table} is a comparison of time complexity and number of rounds that the
%documents need to be processed (each time they get smaller amount of document,
%though). Let $N$ denote the number of total urls collected, $M$ denoete the
%number of cloaking websites. The time complexity is \XXX{Plot}.

\begin{table*}[t]
  \centering
  \begin{center}
    \begin{tabularx}{1\textwidth}{c|c|c|c|c|c}
    \toprule
    \multicolumn{6}{c} {$W$: number of words, $T$: number of tags,
    $L$: number of links} \\
    \midrule
      Methods & Features & Complexity & Performance &
      Efficiency & Deployment \\
      \hline
      Najork ~\cite{najork2005system} & terms, links, tags 
      & $O(W^2 + L^2 + T^2)$ & medium & low & Server \\
      Term \& Link Diff ~\cite{wu2005cloaking}       & terms and links & $O(W^2 + L^2)$
      & low & low & Server  \\
      Wu \& Davison ~\cite{wu2006detecting} & terms, links, tags & $O(W^2 + L^2 + T^2)$ & 
      high & low &  Server \\
      CloakingScore ~\cite{chellapilla2006improving} & terms &   $O(W^2 +
      L^2 + T^2)$ & high & low & Server  \\
      Click-Through ~\cite{wang2006detecting} & click-through  & 
      $O(W^2 + L^2 + T^2)$ & high & low & Server  \\
      TagDiff ~\cite{lin2009detection} & tags  &    $O(T^2)$  & medium &
      high &  Server  \\
      Cloaker and Dagger ~\cite{wang2011cloak} & terms, tags  &  $O(W^2 +
      T^2)$ & high & medium & Server \\
      Hybrid Detection ~\cite{deng2013uncovering} & terms, links, tags &
      $O(W^2 + T^2)$ & high & low & Server \\
      SWM & terms, tags & $O(W + T)$ & high & high & Server and User \\
      \bottomrule
    \end{tabularx}
  \end{center}
  \caption{Comparison of cloaking detection methods}
  \label{comparison}
\end{table*}






\subsection{Server-based Deployment}
In server-based cloaking detection system,
we first collect targeted search terms includes commecial terms, cloaking oriented terms and hot trend words.
Using these search terms, the system retrieves the search results from the search engines for seven times.
First, the system disguises as normal user by using normal user agent. Next, the
system disguises as Google crawler by using the Google Agent and visit landing
pages obtained from the first visit. With this data that
is crawled from Google view, the system uses the simhash method to model the websites.  
The system compares simhash value from user view and
website model learned from Google view. From the comparsion result, the system judges if the website is cloaking
or not. 
Server based cloaking detection system has pros and cons. The deployment of server based cloaking dectetion system
is pratical and could be deployed
easily. The tradeoff of easy deployment is inefficient in IP cloaking detection. Usually, the servers IP addresses
are in a range,  scammers could find this range and serve benign content to crawlers. One solution is buying numerous IP addresses
from ISP providers and distributing IP addresses as similar to real user distribution. This increases cloaking detection cost. In
addition, distributing IP addresses as users is hard. Further, server-based cloaking detection system is infeasiable
to detect cloaking in search engine marketing(SEM). As we mentioned, using the crawlers to visit websites in SEM field
increases the advertisement cost of websites. Moreover, different websites has
different changing periods. For example, {\it www.yahoo.com} 
updates fast, but {\it www.apple.com} is relatively slow. Identifying different 
crawling periods for different websites is difficult. 

\subsection{Crowdsourcing-based Deployment}
Crowdsourcing cloaking detection system includes user-side and server-side component. In user side, users needs to install the cloaking
detection extension in their browers. While users click search results and view
websites, the extension computes the simhash value based on 
website content and layouts. After calculation, extension packs URL and simhash
value and sends to server. Server passively receives $(URL, simhash value)$
pairs from users. Server also utilizes crawlers to extract content several times
from this URL from search engine view (crawler uses Google bot agents).
Server uses these extracted content to model the website. Through comparing
$(URL, simhash value)$ pair from users and crawler view,
server could decide if the website is cloaking or not. As mentioned in
~\autoref{s:measurement}, incentives behind cloaking includes domain parking
~\cite{vissers2015parking} (suspicious) and
malware distribution. Based on cloaking
detection result, server categorizes cloaking and update blacklists in browsers' extensions.
Updating blacklists and warning users phishing and
malware downloading is users incentive to install our extension.

~\autoref{fig:workflow} explains the proposed crowdsourcing-based cloaking detection framework. 
Next, We discuss the pros and cons for this approach. The first advantage is privacy. Instead of solicting
website content from users, the system solicited a 64 bits simhash value from users. 
From this 64 bits value, system couldn't do reverse engineering to
get original content. In addition, the system could intergrate with 
RAPPOR~\cite{erlingsson2014rappor}, which protect the privacy of URL.
Because the workflow is similar to safe browsing API ~\cite{rajab2013camp} , we argue that this can be easily
extended in current framework, the only different is that (URL, simhash value)
pair is processed through RAPPOR to achieve anonymity.

In addition, the crowdsouring deployment introduces low traffic. Browsers only send a 64 bits value for
each URL. This won't jam traffic. Further, the crowdsoucing deployment 
wouldn't affect the model of SEM. Each click on advertisements of search engines is
from real users instead of crawlers. Advertisers don't need to pay extra money for cloaking detection. 


%Employ RAPPOR~\cite{erlingsson2014rappor} to provide user privacy gurantee.
%
%pros: 	1.privacy 2.Low traffic. 3.SEM 4.Distributed computation 
%5. Remove the need to do redirect cloaking detection, leveraging the feature
%that the end goal of attackers is to reach user
%6. could decide crawl period passively based on user clicks, data received are
%based on real user’s clicks, say, website traffic
%
%cons: user incentives. 
%Solution: Plugin to detect suspicious websites. API


\begin{figure}[t]
  \centering
  \includegraphics[width=.45\textwidth]{fig/crowdsourcing-cloaking-detection-system}
  \caption{Workflow of crowdsource cloaking detection sysetem}
  \label{fig:workflow}
\end{figure}

%The workflow ~\autoref{fig:workflow} is to collect page contents simhash on the user side, and compare
%them to simhash of the same link from ad serving company to find cloaking. When
%the differences of the simhashes are significantly large, the page is marked
%cloaking. We generate two simhash for page content and structure respectively.
%Intuition behind this is, simhash difference between different sites are larger
%than different visits of the same site. We build a two-phase system to detect
%cloaking: cluster learning phase, and cloaking detection phase. In the cluster
%learning phase, an ad company visit urls and generate simhash from its content
%with its owned IP, and learn pattern and distribution of the simhashes, i.e.
%simhash-based website model. In the cloaking detection phase, the ad company
%collects simhash from its users. Compare them with learned patterns, return
%cloaking score or mismatch
%


\section{Acknowledgments}

A polite author always includes acknowledgments.  Thank everyone,
especially those who funded the work. 

\section{Availability}

It's great when this section says that MyWonderfulApp is free software, 
available via anonymous FTP from

\begin{center}
{\tt ftp.site.dom/pub/myname/Wonderful}\\
\end{center}

Also, it's even greater when you can write that information is also 
available on the Wonderful homepage at 

\begin{center}
{\tt http://www.site.dom/\~{}myname/SWIG}
\end{center}

Now we get serious and fill in those references.  Remember you will
have to run latex twice on the document in order to resolve those
cite tags you met earlier.  This is where they get resolved.
We've preserved some real ones in addition to the template-speak.
After the bibliography you are DONE.

{\footnotesize \bibliographystyle{acm}
%\bibliography{../common/bibliography}}
\bibliography{ruian,weiren}}

\theendnotes

\end{document}







