\section{Conclusion}
\label{s:conclusion}

Cloaking is an ongoing war between inspectors and cloakers. 
In the field of SEO and SEM, cloakers have different incentives for cloaking,
i.e. monetizing traffic versus providing illegal service.
The two challenges in cloaking deteciton enables cloaking to be popular in the
wild: revealing cloaked content and diffrentiate dynamics from cloaking.
This work proposes Simhash-based Website Model to address these challenges and
achieves 97\% true positive rate at a false positive rate of 0.3\%.  

A great advantage of this
solution compared to past approaches is efficiency and ability to be deployed as
user plugin. By crowdsourcing user views, both challenges are addressed
elegantly. While we are not able to deploy
the crowdsourcing model, we implement the data collection as browser plugin and
encourage search engine or other inspectors to adopt this solution.

%Previous work on simhash focuses on using simhash to find near duplicates, in
%contrast, this work leverage the assuption that content from same website are
%likely to be near duplicates and use simhash to learn clusters and model website
%dynamics.



%design is suitabl
%
%Current solution, multiple pass of data, doesn’t work, and hard to
%handle various types of cloaking.
%Our model works because …
%


