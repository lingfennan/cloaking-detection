\section{Background}
\label{s:background}
\subsection{An Example}
This is an example of cloaking. Search engine cloaking. Show viagra.
Advertisement cloaking. Show essay writing.


\subsection{Cloaking Detection}


Talk about cloaking detection related work here.

\subsection{Simhash}
Talk about simhash related work here. \\
Simhash~\cite{charikar2002similarity}  is a locality sensitive hash algorithm.
It is widely used by search engine to detect near duplicate of websites.



Some embedded literal typeset code might 
look like the following :

{\tt \small
  \begin{verbatim}
  int wrap_fact(ClientData clientData,
  Tcl_Interp *interp,
  int argc, char *argv[) {
    int result;
    int arg0;
    if (argc != 2) {
      interp->result = ``wrong # args'';
      return TCL_ERROR;
    }
    arg0 = atoi(argv[1]);
    result = fact(arg0);
    sprintf(interp->result,``%d'',result);
    return TCL_OK;
  }
  \end{verbatim}
}

Now we're going to cite somebody.  Watch
for the cite tag.
Here it comes~\cite{Chaum1981,Diffie1976}.
The tilde character (\~{})
in the source means a non-breaking space.
This way, your reference will
always be attached to the word that preceded it,
instead of going to the
next line.
