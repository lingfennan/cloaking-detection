\section{Measurement}
\label{s:measurement}

\begin{table*}[t]
  \centering
  \begin{center}
    \begin{tabularx}{1\textwidth}{c|c|c|c|c|c|c|c|c|c|c|c}
      Category & Pharmacy & Gamble & Loan & TS & PPC & Error & IS & Phishing &
      PD &  Malware & Total\\
      \hline
      Spammy Search & 661 & 1514 & 33 & 28 & 28 & 43 & 122 & 17 & 73 & 20 &
      2491 \\
      Hot Search & 33 & 2 & 26 & 27 & 0 &  2 & 2 & 0 &   3 & 0 & 93\\
      Spammy Ads & 0 & 0 & 0 & 0 & 1 & 0 & 5 & 0 & 0 & 0 & 6\\
      Hot Ads & 0 & 0 & 0 & 4 & 0 &  0 & 6 & 0 & 0 & 0 & 10\\
      \bottomrule
      \multicolumn{10}{c} {TS: Traffic Sale, PPC: Pay-Per-Click, IS: Illegal
      Service, PD: Parking Domain}
    \end{tabularx}
  \end{center}
  \caption{Cloaking Distribution.}
\end{table*}


With the model built in ~\autoref{s:evaluation}, we detect cloaking in
four collected datasets, spammy search, $D_{spam, search}$, hot search,
$D_{hot, search}$, spammy ads $D_{spam, ad}$, hot ads, $D_{hot, ad}$. 
$D_{spam, search}$. From our observations, we categorize cloaking websites into 9 types:
pharmacy, gambling, loan, general traffic sale, pay per click, error page, illegal service,
phishing, parking domain and malware downloading. To better analyze cloaking incentives, 
we divided traffic sale into 4 categories: pharmacy, gambling, loan and general traffic sale. 

\subsection{Cloaking in SEO}

In SEO field, we detect cloaking websites in spammy search and hot search field. In spammy search,
we applied our cloaking detection system on 129393 websites. Cloaking detection system reported 2491
cloaking websites. We manually labeled all reported cloaking websites into categories.
661 websites are cloaking of pharmacy. 1514 websites are cloaking of gambling.
33 websites are cloaking of loan. 28 websites are cloaking of general traffic sale. 28 websites are cloaking
of pay per click. 43 websites are cloaking of error page. 122 websites are cloaking of illegal service. 
17 websites are cloaking of phishing. 73 websites are cloaking of parking domain. 20 websites are cloaking of malware downloading.
In hot search, we applied our cloaking detection system on 25533 websites. Cloaking detection system reported 93
cloaking websites. 33 websites are cloaking of pharmacy. 2 websites are cloaking of gambling.
26 websites are cloaking of loan. 27 websites are cloaking of traffic sale. 2 websites are cloaking of cloaking of
error page. 2 websites are cloaking of illegal service. 3 websites are cloaking of parking domain.

From the data from SEO field, we see that the main goal of cloaking as an SEO technique is to obtain user traffic.
In spammy search, 87.3\%traffic sale cloaking is from pharmacy and gamble area. 





%label total cloaking
%173
%phishing
%78 + 69 (gambling)
%cheat, dishonest behavior
%13
%malware or parking domain
%16

%We have detected.
%We manually examine the results and found \XXX{N} are actually cloaking.
%For SEO and SEM, the cloaking rate is 5\% and 3\% in the dataset collected by
%us.
%
%We see a lower cloaking rate compared to XXX, may be because Google has done
%something to this. However, this problem still remains.
%
%For the United States, web search dataset.
%
%The advertisements are
%before dedup: 4381
%after dedup: 1487


\subsection{Cloaking in SEM}

In SEM field, we deteck cloaking websites in spammy ads and hot ads. In spammy ads,
we applied cloaking detection system on 25533 websites. Cloaking detection system reported 6 cloaking websites.
1 website is cloaking of pay per click. In hot ads, we applied cloaking tection system on 25209 websites.
Cloaking detection system reported 10 cloaking websites. 4 websites are cloaking of traffic sale.
6 websites are cloaking of illegal service. 

In SEM, we have detected 100 cloaking examples out of 10k ads. This percentage
is lower compared to SEO. However, ads are much more important because they
matters, clicks in ads equals money. Most of the detected cloaking are providing
illegal services.

We argue that, previous methods cannot be used to detect SEM cloaking, simply
because performing clicks from search engine side, is ad fraud. In contrast,
SWM simply collects the fuzzy signatures of websites. With privacy guarantee
provided by RAPPOR~\cite{erlingsson2014rappor}, and one-wayness of simhash,
crowdsourcing is an achievable and elegant way to detect cloaking.

%\subsection{Cross Domain Spam Detection}
%In our detection result, we have observed many cloaking cases, where URL are
%completely different, while the content are similar or even the same.
%We argue that, the foundamental reason for them to do this is the low cost of
%getting a new URL and lack of efficient way to detect spammy content.
%Based on this observation, we propose content based blacklist to raise the bar
%for reusing spammy content. This approach leverages the most popular use of
%simhash - near duplicate detection. For spammy pages, in order to evade our
%detection, they not only need to change URL rapidly, but also update their
%content everytime, which can be expensive in their current mode if they want
%every copy their website to be different and have meaningful and stay attractive
%to user.
%


