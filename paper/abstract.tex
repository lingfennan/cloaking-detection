\subsection*{Abstract}
Cloaking has long been used by spammers for the purpose of increasing
the exposure of their websites, serving as a major malicious technique
in search engine optimization. More recently, we have also witnessed a
rising trend of employing cloaking in search engine marketing. The
motivation of cloaking is to hide the true nature of a website by
delivering blatantly different content to users versus web crawlers
such as search engines. Cloaking is a popular technique due to its low
setup cost and the lack of effective and efficient detection methods.

In this paper, we propose Simhash-based Website Model, or SWM, a new
approach to fuzzy hash the content and layout of a website and model
the dynamics of a website.  We apply SWM to cloaking detection, and
our evaluation results show that SWM achieves 97.1\% true positive
rate with 0.3\% false positive rate.  SWM is more efficient compared
with past approaches and can be deployed as browser plugin for
crowdsourcing user views of websites with negligible overhead.  In
order to understand the incentives behind cloaking, we systematically
categorize and review the detected samples. Our analysis shows that
majority of them are abusing search engine with traffic sale, and
4.3\% are malicious or phishing sites.  
%With motivation of both search
%engine and user to combat cloaking, we discuss deployment of SWM in
%cloaking detection and propose a novel model to detect cloaking
%through crowdsourcing with user privacy guarantee.

\subsection*{Keywords}
SEO, SEM, Online Advertising, Cloaking Detection, Similarity Detection
Algorithm

