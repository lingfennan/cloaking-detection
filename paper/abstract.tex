
\subsection*{Abstract}
Cloaking, used by spammers for the purpose of increasing the visiting rates of
their website, as well as to circumvent censorship, has been a challenging
spamming technique to search engines. Recently, there is a rising trend of
employing
cloaking in advertisement spam ~\cite{li2012knowing}.  The motivation of
cloaking 
is to hide the true nature of a website by delivering different semantic
content users versus censors.
Cloaking is popular due to low cost to setup and lack of efficient technique
to detect.

There are two major challenges in cloaking detection (A) revealing the
uncloaked content 
(B) differentiating cloaking from naturally dynamic changes of the same page.
This work employs
simhash-based website model to detect cloaking. The key idea to address
challenge 
(A) is to crowd-source data collection to user. However, collect such
information 
introduces large volume of traffic and violates user privacy. 
This work proposes simhash-based website model to address challenge (B)
with minimum amount of data and guarantee of user privacy.

In order to show the effectiveness of simhash-based website model, we build
a prototype and results 
show that we can achieve more than 95\%? true positive rate with 5\%? false
positive rate.

Precision is 0.9610642439974043, out of 1541.


