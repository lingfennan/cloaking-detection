
\subsection*{Abstract}
Cloaking, used by spammers for the purpose of increasing exposure of
their websites, as well as to circumvent censorship, has been a nostorious
spamming Search Engine Optimization (SEO) technique. Recently, besides SEO, we also
identify a rising trend of employing cloaking in search engine marketing. The motivation of cloaking 
is to hide the true nature of a website by delivering blatantly different 
content to users versus censors. Cloaking is popular due to low setup cost 
and lack of efficient detection technique. 

In this paper, we propose Simhash-based Website Model, or SWM, a new way to
fuzzy hash the content and layout of a
website and model the dynamics of a website. 
We apply SWM in cloaking detection, and achieve comparable accuracy and much
higher efficiency
compared with past approaches. In order to understand the incentives
behind cloaking, we systematically
categorize and review the detected samples, and find 9\% of them are malicious
websites, and 80\% of them are phishing sites. 
With this strong relationship between cloaking and maliciousness of a website,
we discuss deployment of SWM in cloaking detection and
propose a novel model to
detect cloaking through crowdsourcing with user privacy guarantee.

