
\subsection*{Abstract}
Cloaking, used by spammers for the purpose of increasing exposure of
their websites, as well as to circumvent censorship, has been a notorious 
spamming search engine optimization technique. Recently, we also
identify a rising trend of employing cloaking in search engine marketing. The motivation of cloaking 
is to hide the true nature of a website by delivering blatantly different 
content to users versus censors. Cloaking is popular due to low setup cost 
and lack of efficient detection technique. 

In this paper, we propose Simhash-based Website Model, or SWM, a new way to
fuzzy hash the content and layout of a
website and model the dynamics of a website. 
We apply SWM in cloaking detection, and achieve 97.1\% true positive rate with
0.3\% false positive rate. 
SWM is more efficient compared to past approaches and can be deployed as browser
plugin for crowdsourcing user view of websites with negligible overhead.
In order to understand the incentives
behind cloaking, we systematically
categorize and review the detected samples, and find majority of them are
doing abusing search engine with traffic sale, and 4.3\% are 
malicious or phishing sites.
With motivation of both search engine and user to combat cloaking, we discuss
deployment of SWM in cloaking detection and
propose a novel model to
detect cloaking through crowdsourcing with user privacy guarantee.

