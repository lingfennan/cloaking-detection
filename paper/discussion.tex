\section{Discussion}
\label{s:discussion}

In this section, we first compare our work with previous cloaking detection
works, then discuss two kinds of deployment of simhash-based cloaking
detection system: server-based and crowdsoucing. The attack model and robustness
are discussed as well.

\subsection{Efficiency Comparison}
\label{ss:efficiency}
Previous cloaking detection approaches use various feature set from a website,
and the latest work ~\cite{wang2011cloak} use text features, dom features and
search snippet to detect cloaking. Comparing with their model, we achieve
comparable precision and recall (transformed from FPR and TPR). However, a great
advantage of our work compared to past approaches is that, our algorithm is
clean (requires only single pass of website, to get fuzzy signature) and 
can be deployed at user side for crowdsourcing.


Previous cloaking detection approaches are mostly measured in terms of precision
and recall, and the latest work ~\cite{wang2011cloak} , and after transformation from FPR and TPR to precision and recall,
we achieve better performance. However, the great advantage of

Compared with past approaches, we achiev
While we can achieve similar false positve rate and true positive rate compared
to past approaches, we argue that, our approach is much more efficient than past
approaches and is light-weight enough to be deployed on user browser.
\XXX{Table} is a comparison of time complexity and number of rounds that the
documents need to be processed (each time they get smaller amount of document,
though). Let $N$ denote the number of total urls collected, $M$ denoete the
number of cloaking websites. The time complexity is \XXX{Plot}.



For simhash computation, we implements a browser extension which exposes
negligible overhead to user. We 

\XXX{the table comparing our approach with past cloaking detection approaches}
\XXX{the ability of simhash in measuring difference}.


\begin{table*}[t]
  \centering
  \begin{center}
    \begin{tabularx}{1\textwidth}{c|c|c|c|c|c|c}
      Methods & Features & Detection Ability & Complexity & Performance &
      Efficiency & Deployment \\
      \hline
      Spammy Search & 661 & 1514 & 33 & 28 & 28 & 
      2491 \\
      Hot Search & 33 & 2 & 26 & 27 & 0 &  2\\
      Spammy Ads & 0 & 0 & 0 & 0 & 1 &2  \\
      Hot Ads & 0 & 0 & 0 & 4 & 0 &2  \\
      \bottomrule
      \multicolumn{7}{c} {Detection Ability: Repeat Cloaking, IP Cloaking,
      Referer Cloaking, User Agent Cloaking}

    \end{tabularx}
  \end{center}
  \caption{Comparison of cloaking detection methods}
\end{table*}






\subsection{Server-based Cloaking Detetion}
In server-based cloaking detection system,
we first collect targeted search terms includes commecial terms, cloaking oriented terms and hot trend words.
Using these search terms, the cralwer extracts the search results from the search engines for seven times.
In six times, the crawler disguises as Google crawler by using the Google Agent and crawl data. With this data that
is crawled from Google view, the system uses the simhash method to model the websites. The last
time, the crawler disguises as normal user by using user agent. The system compares simhash value from user view and
website model learned from Google view. From the comparsion result, the system judges if the website is cloaking
or not. 
Server based cloaking detection system has pros and cons. The deployment of server based cloaking dectetion system
is pratical and could be deployed
easily. The tradeoff of easy deployment is inefficient in IP cloaking detection. Usually, the servers IP addresses
are in a range,  scammers could find this range and serve benign content to crawlers. One solution is buying numerous IP addresses
from ISP providers and distributing IP addresses as users distribution. This increases cloaking detection cost. In
addition, distributing IP addresses as users distribution is hard. Further, server-based cloaking detection system is infeasiable
to detect cloaking in search engine marketing(SEM). As we mentioned, using the crawlers to visit websites in SEM field
increases the advertisement cost of websites. Moreover, different websites has different changing periods. For example, Yahoo website
updates very quickly. Apple website updates slowly. Finding different crawling periods for different websites is difficult. 

\subsection{Crowdsourcing-based Cloaking Detection}
Crowdsourcing cloaking detection system includes user-side and server-side component. In user side, users needs to install the cloaking
detection extension in their browers. While users click search results and view websites, the extention calculate the simhash value based on 
website content and layouts. After calculation, extension packs URL and simhash value and sends to server. Server passively receives (URL, simhash value)
pairs from users. Server also utilizes crawlers to extract content several times in this URL from search engine view (crawler uses Google bot agents).
Server uses these extract content to model the website. From the comparison result between (URL, simhash value) from users and website model,
server could decide if the website is cloaking or not. From previous study, cloaking includes phishing and malware downloading. Based on cloaking
detection result, server categorizes cloaking and update blacklists in browsers' extensions. Updating blacklists and warning users phishing and
malware downloading is users incentive to install our extension.

Previously, we describe our crowdsoucing workflow. Next, We discuss the pros and cons for this approach. The first advantage is privacy. Instead of solicting
website content from users, the system solicited a 64 bits simhash value from users. From this 64 bits value, system couldn't do reverse engineering to
get original content. In addition, the system could intergrate with RAPPOR~\cite{erlingsson2014rappor}, which protect the privacy of URL.
Because the workflow is similar to safe browsing API ~\cite{rajab2013camp} , we argue that this can be easily
extended in current framework, the only different is that (URL, simhash value)
pair is processed through RAPPOR to achieve anonymity.

In addition, the crowdsouring deployment introduces low traffic. Browsers only send a 64 bits value for
each URL. This won't jam traffic. Further, the crowdsoucing deployment wouldn't affect the model of SEM. Each click on advertisements of search engines is
from real users instead of crawlers. Advertisers don't need to pay extra money for cloaking detection. 


Employ RAPPOR~\cite{erlingsson2014rappor} to provide user privacy gurantee.

pros: 	1.privacy 2.Low traffic. 3.SEM 4.Distributed computation 
5. Remove the need to do redirect cloaking detection, leveraging the feature
that the end goal of attackers is to reach user
6. could decide crawl period passively based on user clicks, data received are
based on real user’s clicks, say, website traffic
%
%cons: user incentives. 
%Solution: Plugin to detect suspicious websites. API


\begin{figure}[t]
  \centering
  \includegraphics[width=.45\textwidth]{fig/crowdsourcing-cloaking-detection-system}
  \caption{Workflow of crowdsource cloaking detection sysetem}
  \label{fig:workflow}
\end{figure}


The workflow ~\autoref{fig:workflow} is to collect page contents simhash on the user side, and compare
them to simhash of the same link from ad serving company to find cloaking. When
the differences of the simhashes are significantly large, the page is marked
cloaking. We generate two simhash for page content and structure respectively.
Intuition behind this is, simhash difference between different sites are larger
than different visits of the same site. We build a two-phase system to detect
cloaking: cluster learning phase, and cloaking detection phase. In the cluster
learning phase, an ad company visit urls and generate simhash from its content
with its owned IP, and learn pattern and distribution of the simhashes, i.e.
simhash-based website model. In the cloaking detection phase, the ad company
collects simhash from its users. Compare them with learned patterns, return
cloaking score or mismatch

\subsection{Robustness}
Attack model: Attacker has the simhash of the content that user is viewing, and
he want to find out the content.
Defense: Because simhash has the property of one-wayness, attacker could not get
the content. More formally speaking, pre-image attack fails. This is actually
straightforward because when processing each feature, we apply hash functions
which has the property of one-wayness.

Attack Model: Attacker get the URL and corresponding simhash of the page that
user is viewing, he visit the same page and want to find some sensitive
information for the user on that page, e.g. name, numbers.
Defense: Let x and y denote two sets, h denote simhash function, the probability
that simhash of x and y collides is cosine similarity between x and y, and can
be expressed by PrhF[h(x)=h(y)]=sim(x,y). When size of x and y are comparably
large, say, 100, one word difference between x and y will result in the same
simhash (collision). Therefore, attacker could not do dictionary attack on
sensitive field of that page. Because feature set is large and difference in
several features result in the same simhash.

Attackers (bad advertisers who get to know what we are doing) may
(1) Maintain dom structure, but show different text
In this case, we need to come up with a good way to combine dom simhash score
and text simhash score, to catch this kind of cloaking. (intersection of dom
mismatch and text mismatch may be evaded by this approach)
(2) Deliver bad content to Google periodically
In this case, we can get content from Google, and this is technically cloaking
anymore. As long we can force the advertiser to deliver bad content to Google,
then we are winning this cloaking game. At Google side, we can have other
independent approaches to distinguish and detect bad content and good content.
(3) To reduce workload, ad serving company have to sample the ads at some
probability, then cloakers may try to detect whether we are sampling and deliver
good content if we are sampling or simply cloak at some probability, to reduce
the risk of being caught.
In this case, we need to prove, (a) when we sample at low probability, the
cloakers will still be caught even if they cloak randomly, e.g. 10\% probability
(10\% users will definitely be caught). (b) it is hard for the cloakers to find
when we are sampling.



